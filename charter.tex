\documentclass[
11pt, % The default document font size, options: 10pt, 11pt, 12pt
%codirector, % Uncomment to add a codirector to the title page
]{charter} 


% El títulos de la memoria, se usa en la carátula y se puede usar el cualquier lugar del documento con el comando \ttitle
\titulo{Monitoreo de red de sensores Bluetooth
en invernaderos} 

% Nombre del posgrado, se usa en la carátula y se puede usar el cualquier lugar del documento con el comando \degreename
\posgrado{Carrera de Especialización en Sistemas Embebidos} 
%\posgrado{Carrera de Especialización en Internet de las Cosas} 
%\posgrado{Carrera de Especialización en Inteligencia Artificial}
%\posgrado{Maestría en Sistemas Embebidos} 
%\posgrado{Maestría en Internet de las cosas}

% Tu nombre, se puede usar el cualquier lugar del documento con el comando \authorname
% IMPORTANTE: no omitir titulaciones ni tildación en los nombres, también se recomienda escribir los nombres completos (tal cual los tienen en su documento)
\autor{Ing. Laura Andrea Moreno Rodríguez}

% El nombre del director y co-director, se puede usar el cualquier lugar del documento con el comando \supname y \cosupname y \pertesupname y \pertecosupname
\director{Ing. Federico Roux}
\pertenenciaDirector{Globant} 
\codirector{} % para que aparezca en la portada se debe descomentar la opción codirector en los parámetros de documentclass
\pertenenciaCoDirector{FIUBA}

% Nombre del cliente, quien va a aprobar los resultados del proyecto, se puede usar con el comando \clientename y \empclientename
\cliente{Pablo Lodetti}
\empresaCliente{Wentux Tecnoagro}
 
\fechaINICIO{11 de marzo de 2025}		%Fecha de inicio de la cursada de GdP \fechaInicioName
\fechaFINALPlan{22 de abril de 2025} 	%Fecha de final de cursada de GdP
\fechaFINALTrabajo{noviembre 2025}	%Fecha de defensa pública del trabajo final


\begin{document}

\maketitle
\thispagestyle{empty}
\pagebreak


\thispagestyle{empty}
{\setlength{\parskip}{0pt}
\tableofcontents{}
}
\pagebreak


\section*{Registros de cambios}
\label{sec:registro}


\begin{table}[ht]
\label{tab:registro}
\centering
\begin{tabularx}{\linewidth}{@{}|c|X|c|@{}}
\hline
\rowcolor[HTML]{C0C0C0} 
Revisión & \multicolumn{1}{c|}{\cellcolor[HTML]{C0C0C0}Detalles de los cambios realizados} & Fecha      \\ \hline
0      & Creación del documento                                 &\fechaInicioName \\ \hline
1      & Se completa hasta el punto 5 inclusive                & {19} de {marzo} de 2025 \\ \hline
%2      & Se completa hasta el punto 9 inclusive
%		  Se puede agregar algo más \newline
%		  En distintas líneas \newline
%		  Así                                                    & {día} de {mes} de 202X \\ \hline
%3      & Se completa hasta el punto 12 inclusive                & {día} de {mes} de 202X \\ \hline
%4      & Se completa el plan	                                 & {día} de {mes} de 202X \\ \hline

% Si hay más correcciones pasada la versión 4 también se deben especificar acá

\end{tabularx}
\end{table}

\pagebreak



\section*{Acta de constitución del proyecto}
\label{sec:acta}

\begin{flushright}
Buenos Aires, \fechaInicioName
\end{flushright}

\vspace{2cm}

Por medio de la presente se acuerda con la \authorname\hspace{1px} que su Trabajo Final de la \degreename\hspace{1px} se titulará ``\ttitle'' y consistirá en la implementación de un protocolo de comunicación basado en Bluetooth Mesh para interconectar diferentes sensores dentro de un invernadero, así como el desarrollo de un servidor web embebido en el dispositivo central para optimizar el monitoreo y control local de la red. El trabajo tendrá un presupuesto preliminar estimado de 600 horas y un costo estimado de {\$ 9600 USD}, con fecha de inicio el \fechaInicioName\hspace{1px} y fecha de presentación pública en \fechaFinalName.

Se adjunta a esta acta la planificación inicial.

\vfill

% Esta parte se construye sola con la información que hayan cargado en el preámbulo del documento y no debe modificarla
\begin{table}[ht]
\centering
\begin{tabular}{ccc}
\begin{tabular}[c]{@{}c@{}}Dr. Ing. Ariel Lutenberg \\ Director posgrado FIUBA\end{tabular} & \hspace{2cm} & \begin{tabular}[c]{@{}c@{}}\clientename \\ \empclientename \end{tabular} \vspace{2.5cm} \\ 
\multicolumn{3}{c}{\begin{tabular}[c]{@{}c@{}} \supname \\ Director del Trabajo Final\end{tabular}} \vspace{2.5cm} \\
\end{tabular}
\end{table}




\section{1. Descripción técnica-conceptual del proyecto a realizar}
\label{sec:descripcion}
Este proyecto surge como una necesidad de la empresa {\empclientename}, quien lo ha propuesto dentro del programa de vinculación con empresas de la {\degreename}. La empresa se dedica a la fabricación y comercialización de diversos dispositivos para la automatización de salas de cultivo, siendo los sensores de temperatura, humedad y Co2, algunos de sus productos más destacados. Actualmente, los sensores fabricados por {\empclientename} utilizan el microcontrolador ESP32-C3 como procesador central y transmiten los datos de medición a través de la conexión Wi-Fi del usuario final. Sin embargo, este enfoque limita la instalación de los sensores en las salas de cultivo, ya que todos deben estar dentro del alcance de la red Wi-Fi para funcionar correctamente.

El objetivo principal de este proyecto es aprovechar las capacidades Bluetooth del ESP32-C3 para implementar una solución de transmisión de datos mediante una red de sensores Bluetooth Mesh. Esta tecnología permitirá superar la limitación de cobertura Wi-Fi, al proporcionar una red descentralizada donde cada sensor puede transmitir datos a través de otros nodos, extendiendo el alcance y mejorando la confiabilidad del sistema. Además, se desarrollará un servidor web embebido en el sensor central que facilitará la configuración y monitoreo de la red, optimizando la gestión y operación del sistema para el usuario final.

Es importante señalar que el desarrollo y las pruebas del sistema no se llevarán a cabo con los dispositivos comerciales de {\empclientename}. En su lugar, se emplearán ESP32-C3 adquiridos específicamente para este proyecto, ya que son suficientes para desarrollar y validar la prueba de concepto. El enfoque principal se centrará en el diseño e implementación de la comunicación Bluetooth Mesh y el servidor web local, dejando la integración del código con los dispositivos reales para una fase posterior a cargo de {\empclientename}. Esto significa, que no se requiere acceso al hardware o software de la empresa, pero sí acompañamiento e información oportuna para poder simular los datos que sean necesarios y crear un entorno de desarrollo adecuado. Por otra parte, el cliente tendrá acceso completo al software desarrollado y la licencia para integrarlo en sus dispositivos desde el inicio del proyecto.

La motivación de este proyecto radica en la oportunidad de aplicar los conocimientos adquiridos durante la {\degreename}, a la vez que continúo desarrollando habilidades clave en el desarrollo de firmware. Los ESP32-C3 y la tecnología Bluetooth Mesh son ampliamente utilizados en diversos sistemas de internet de las cosas, lo que hace que este proyecto sea valioso tanto para el aprendizaje personal como para mi futuro profesional. Además, resulta especialmente gratificante contribuir al crecimiento de una micro empresa, ayudándola a agregar valor a sus productos y a mejorar su competitividad en el mercado.

Una red de sensores Bluetooth Mesh es un sistema de comunicación inalámbrica en el que múltiples dispositivos (nodos) se interconectan para formar una red descentralizada de amplio alcance. En este tipo de red, cada nodo retransmite los datos recibidos, lo que permite extender la cobertura de comunicación más allá del alcance de un único dispositivo. En el contexto de sensores para invernaderos, en donde cada sensor es un nodo de la red, esta arquitectura posibilita la transmisión eficiente de los datos entre los nodos hasta alcanzar un nodo central. Esto mejora el consumo energético, reduce la infraestructura necesaria y facilita la escalabilidad del sistema.

La figura \ref{fig:diagBloquesBleMesh} ilustra el principio de comunicación en una red Bluetooth Mesh a implementar en este proyecto, donde los nodos, que en este caso corresponden a los sensores en el invernadero, colaboran para transmitir los datos recopilados de manera eficiente hasta llegar al nodo central. La diferencia entre un nodo y un nodo central radica principalmente en la configuración asignada a cada dispositivo durante la instalación de la red.

En términos de hardware, todos los nodos están basados en el ESP32-C3, como se muestra en la figura \ref{fig:diagBloquesEsp32}, que representa una versión simplificada de la arquitectura interna de cada sensor. La única diferencia funcional entre un nodo y el nodo central es que este último incorporará un servidor web embebido, al cual el usuario final podrá acceder para visualizar los datos recopilados por la red. Como resultado, el nodo central es el único que requiere una conexión estable a Wi-Fi, permitiendo el acceso a la interfaz de monitoreo desde cualquier dispositivo conectado a la misma red, como una computadora o un teléfono inteligente.

\begin{figure}[htpb]
\centering 
\includegraphics[width=.95\textwidth]{./Figuras/Diagrama-ble-mesh.png}
\caption{Diagrama de red Bluethtoh Mesh.}
\label{fig:diagBloquesBleMesh}
\end{figure}

\begin{figure}[htpb]
\centering 
\includegraphics[width=.42\textwidth]{./Figuras/Diagrama-esp32-c3.png}
\caption{Diagrama en bloques de cada nodo en la red Bluethooth Mesh.}
\label{fig:diagBloquesEsp32}
\end{figure}

La implementación de la red Bluetooth Mesh se basará en el SDK proporcionado por Espressif, fabricante del microcontrolador ESP32-C3, el cual ya incluye los protocolos necesarios para la provisión, enrutamiento y gestión de los nodos. No obstante, el enfoque de este trabajo irá más allá de la implementación básica, ya que se personalizará la solución para cumplir con los requisitos específicos del cliente. Se desarrollará una librería modular, que facilitará la integración de la red Mesh en los sensores de {\empclientename}, y un servidor web local, funcionalidad que no está contemplada en la solución estándar de Espressif. Este servidor permitirá la gestión de la red de sensores en tiempo real a través de una interfaz web intuitiva, optimizando la experiencia del usuario y brindando un mayor control sobre el sistema.

El desarrollo de este proyecto presenta varios desafíos, especialmente en la gestión eficiente de la comunicación entre nodos, asegurando una transmisión de datos estable en un entorno propenso a interferencias. Además, es fundamental optimizar el consumo energético de los dispositivos, ya que los sensores deben operar de manera eficiente sin afectar su autonomía. Otro reto importante será la integración del protocolo de comunicación con el servidor web local, garantizando que la configuración y monitoreo de la red sean intuitivos para el usuario final. Finalmente, el código deberá ser modular y adaptable para que {\empclientename pueda integrarlo en sus dispositivos reales sin modificaciones estructurales significativas, lo que requerirá una arquitectura bien diseñada y documentada para facilitar futuras expansiones.


\section{2. Identificación y análisis de los interesados}
\label{sec:interesados}

\begin{table}[ht]
%\caption{Identificación de los interesados}
%\label{tab:interesados}
\begin{tabularx}{\linewidth}{@{}|l|X|X|l|@{}}
\hline
\rowcolor[HTML]{C0C0C0} 
Rol           & Nombre y Apellido & Organización 	& Puesto 	\\ \hline
Cliente       & \clientename      &\empclientename	& Responsable técnico	\\ \hline
Responsable   & \authorname       & FIUBA        	& Alumno 	\\ \hline
Orientador    & \supname	      & \pertesupname 	& Director del Trabajo Final \\ \hline
\end{tabularx}
\end{table}

\begin{itemize}
    \item \textbf{Cliente:} El señor Pablo Lodetti es el fundador de la empresa {\empclientename} y el responsable técnico de sus productos. Junto a él se definieron el alcance del proyecto y los entregables esperados.  
    \item \textbf{Orientador:} El {\supname} es especialista en Sistemas Embebidos y brindará orientación tanto en la arquitectura del sistema a implementar como en el desarrollo del firmware embebido. 
\end{itemize}

\section{3. Propósito del proyecto}
\label{sec:proposito}

Desarrollar un sistema de comunicación basado en Bluetooth Mesh para la interconexión de sensores para invernaderos, utilizando el microcontrolador ESP32-C3, eliminando la dependencia de la cobertura Wi-Fi y ampliando el alcance de transmisión de datos. Esto permitirá una mayor flexibilidad en la instalación de los sensores, optimizando la recolección y monitoreo de la información ambiental. Además, se implementará un servidor web local en el nodo central para la configuración y supervisión de la red, mejorando la eficiencia y facilidad de uso del sistema para el cliente final.

\section{4. Alcance del proyecto}
\label{sec:alcance}

Este proyecto incluye: 

\begin{itemize}
\item Implementación de una red de sensores Bluetooth Mesh utilizando el microcontrolador ESP32-C3, basada en el SDK de Espressif.
\item Desarrollo de una librería modular para la comunicación Bluetooth Mesh, permitiendo su fácil integración en los dispositivos del cliente.
\item Implementación de un nodo central con servidor web local, que actuará como punto de recopilación de datos y permitirá la visualización y configuración de la red de sensores.
\item Simulación de datos de sensores en los microcontroladores ESP32-C3 adquiridos para el proyecto, en lugar de utilizar los sensores reales de la empresa.
\item Optimización del consumo energético de los nodos sensores para mejorar su autonomía dentro de la red.
\item Diseño de una arquitectura adaptable, asegurando que el firmware desarrollado pueda ser integrado posteriormente en los sensores reales sin modificaciones estructurales significativas.
\item Desarrollo de una interfaz web intuitiva para el monitoreo y configuración de la red de sensores, accesible a través del nodo central.
\item Validación del sistema en condiciones simuladas, asegurando su funcionamiento antes de la integración con los dispositivos del cliente.
\item Documentación técnica del proyecto, incluyendo la descripción de la arquitectura, instrucciones de integración y uso de la librería y servidor web.

\end{itemize}



Este listado aclara qué aspectos quedan fuera del alcance del proyecto:

\begin{itemize}
\item Uso de los sensores reales de la empresa: Se trabajará con microcontroladores ESP32-C3 adquiridos para el proyecto, simulando los datos de medición en lugar de utilizar los sensores comerciales de {\empclientename}.
\item Desarrollo de hardware personalizado: El proyecto no incluye el diseño o modificación del hardware de los sensores actuales de la empresa, sino únicamente el desarrollo del software.
\item Integración final con los dispositivos comerciales: La implementación en los sensores reales será responsabilidad del cliente, quien podrá integrar la librería desarrollada.
\item Soporte para otras tecnologías de comunicación: Se trabajará exclusivamente con Bluetooth Mesh y Wi-Fi en el nodo central, sin incluir otros protocolos como LoRa, Zigbee o LTE.
\item Acceso a software privativo de la empresa: No se requerirá acceso al firmware actual de los sensores comerciales de {\empclientename}.
\item Implementación de seguridad avanzada: La seguridad de la red Bluetooth Mesh se manejará con las características estándar del SDK de Espressif, sin incluir desarrollos adicionales en cifrado o autenticación avanzada.
\item Almacenamiento en la nube o acceso remoto: El servidor web será local y accesible solo dentro de la red Wi-Fi donde esté conectado el nodo central. No se incluirá conectividad con servicios en la nube ni acceso remoto externo.
\item Soporte para aplicaciones móviles: La visualización y configuración se realizará a través de la interfaz web del nodo central, sin el desarrollo de una aplicación móvil dedicada.
\item Mantenimiento o soporte post-proyecto: No se incluye una fase de soporte o mantenimiento continuo una vez entregado el código y la documentación.
\item Pruebas de alcance, latencia y consumo energético en diferentes escenarios de implementación.
\item Evaluación del rendimiento en un entorno de invernadero real.
\end{itemize}

\section{5. Supuestos del proyecto}
\label{sec:supuestos}

\begin{itemize}
\item Disponibilidad de tiempo
\subitem La \authorname cuenta con el tiempo suficiente para completar el desarrollo del proyecto dentro de un plazo de siete meses, con una dedicación promedio de 22 horas por semana. 
\subitem No habrá interrupciones significativas en el desarrollo del proyecto debido a cambios en la disponibilidad del equipo de trabajo.
\subitem Se espera contar con la colaboración de {\clientename} y {\empclientename} para responder consultas técnicas o aclaraciones necesarias durante el desarrollo.

\item Disponibilidad de recursos materiales
\subitem Se dispone de al menos 4 microcontroladores ESP32-C3 para el desarrollo y pruebas del sistema.
\subitem Se cuenta con acceso a herramientas de desarrollo adecuadas, incluyendo computadoras, compiladores, depuradores y hardware de prueba.
\subitem Se dispone de un entorno adecuado para realizar pruebas de conectividad Bluetooth Mesh en condiciones similares a un invernadero.
\subitem Se asume que {\empclientename} proporcionará la información técnica necesaria sobre sus sensores y sus requisitos específicos de integración.

\item Factibilidad técnica
\subitem El SDK de Espressif para Bluetooth Mesh funciona correctamente y cumple con las necesidades del proyecto sin requerir modificaciones profundas.
\subitem La red Bluetooth Mesh tendrá un rendimiento adecuado para transmitir los datos de los sensores dentro de un invernadero típico, sin interferencias significativas.
\subitem La implementación del servidor web en el nodo central del sistema será viable y permitirá la visualización y gestión de la red en tiempo real.
\subitem La integración de la librería desarrollada con los sensores de {\empclientename} podrá realizarse sin cambios estructurales en su firmware actual.

\item Condiciones externas
\subitem No habrá cambios regulatorios o restricciones tecnológicas que afecten la implementación de Bluetooth Mesh en el entorno del cliente.
\subitem No se prevé escasez de microcontroladores ESP32-C3 u otros componentes electrónicos esenciales durante el desarrollo del proyecto.
\subitem No habrá fluctuaciones significativas en costos de hardware o herramientas necesarias que impacten la viabilidad del proyecto.
\end{itemize}

\section{6. Requerimientos}
\label{sec:requerimientos}

\begin{consigna}{red} % ELIMINAR \begin{consigna}{red} y \end{consigna}{red} en las secciones que vayan completando para cada entrega parcial.
Los requerimientos deben enumerarse y de ser posible estar agrupados por afinidad, por ejemplo:

\begin{enumerate}
	\item Requerimientos funcionales:
		\begin{enumerate}
			\item El sistema debe...
			\item Tal componente debe...
			\item El usuario debe poder...
		\end{enumerate}
	\item Requerimientos de documentación:
		\begin{enumerate}
			\item Requerimiento 1.
			\item Requerimiento 2 (prioridad menor)
		\end{enumerate}
	\item Requerimiento de testing...
	\item Requerimientos de la interfaz...
	\item Requerimientos interoperabilidad...
	\item etc...
\end{enumerate}

Leyendo los requerimientos se debe poder interpretar cómo será el proyecto y su funcionalidad.

Indicar claramente cuál es la prioridad entre los distintos requerimientos y si hay requerimientos opcionales. 

\textbf{¡¡¡No olvidarse de que los requerimientos incluyen a las regulaciones y normas vigentes!!!}

Y al escribirlos seguir las siguientes reglas:
\begin{itemize}
	\item Ser breve y conciso (nadie lee cosas largas). 
	\item Ser específico: no dejar lugar a confusiones.
	\item Expresar los requerimientos en términos que sean cuantificables y medibles.
\end{itemize}

\end{consigna} % ELIMINAR \begin{consigna}{red} y \end{consigna}{red} en las secciones que vayan completando para cada entrega parcial.

\section{7. Historias de usuarios (\textit{Product backlog})}
\label{sec:backlog}

\begin{consigna}{red}
Descripción: en esta sección se deben incluir las historias de usuarios y su ponderación (\textit{history points}). Recordar que las historias de usuarios son descripciones cortas y simples de una característica contada desde la perspectiva de la persona que desea la nueva capacidad, generalmente un usuario o cliente del sistema. La ponderación es un número entero que representa el tamaño de la historia comparada con otras historias de similar tipo.

Se debe indicar explícitamente el criterio para calcular los \textit{story points} de cada historia.

El formato propuesto es: 
\begin{enumerate}
\item ``Como [rol] quiero [tal cosa] para [tal otra cosa]."

\textit{Story points}: 8 (complejidad: 3, dificultad: 2, incertidumbre: 3)
\end{enumerate}
\end{consigna}

\section{8. Entregables principales del proyecto}
\label{sec:entregables}

%Los siguientes son los entregables de este proyecto: 
%\begin{itemize}
%	\item Diseño e Implementación del Protocolo de Comunicación Bluetooth Mesh
%		\subitem Desarrollo del software para la comunicación entre sensores utilizando Bluetooth Mesh.
%		\subitem Configuración y optimización de los parámetros de la red para garantizar eficiencia y estabilidad.
%1		\subitem Documentación técnica del protocolo, incluyendo diagramas de arquitectura y flujo de datos.

%	\item Desarrollo del Firmware para los Sensores ESP32-C3
%		\subitem Implementación del firmware en microcontroladores ESP32-C3 independientes, donde se simularán los datos de medición.
%		\subitem Desarrollo del protocolo de comunicación Bluetooth Mesh en forma de una librería modular, facilitando su integración en los sensores comerciales de {\empclientename}.
%	\subitem Código estructurado y documentado para permitir futuras modificaciones y escalabilidad.
%Optimización del consumo energético en los sensores.
	
%	\item Servidor Web Local para Monitoreo y Configuración
	%	\subitem Desarrollo de una aplicación web embebida para la visualización de datos en tiempo real.
	%	\subitem Implementación de una interfaz gráfica intuitiva para la configuración de la red y los sensores.
	%	\subitem Almacenamiento de datos y configuración en la memoria EEPROM del dispositivo central.
		

	%\item Integración y Pruebas del Sistema Completo
	%	\subitem Pruebas de integración entre sensores, red Bluetooth Mesh y servidor web.
		
	%\item Documentación y Manuales
	%	\subitem Manual de usuario para la configuración y operación del sistema.
	%	\subitem Documentación técnica detallada para el mantenimiento y futuras mejoras.
	%	\subitem Guía de instalación y despliegue del sistema en invernaderos.
		
	%\item Presentación Final del Proyecto
%		\subitem Informe final con todos los resultados y conclusiones del desarrollo.
	%	\subitem Presentación técnica para {\empclientename} y el programa de vinculación empresarial.
%		\subitem Entrega del código fuente, documentación y archivos de configuración.	
%\end{itemize}

\begin{consigna}{red}
Los entregables del proyecto son (ejemplo):

\begin{itemize}
	\item Manual de usuario.
	\item Diagrama de circuitos esquemáticos.
	\item Código fuente del firmware.
	\item Diagrama de instalación.
	\item Memoria del trabajo final.
	\item etc...
\end{itemize}
\end{consigna}

\section{9. Desglose del trabajo en tareas}
\label{sec:wbs}

\begin{consigna}{red}
El WBS debe tener relación directa o indirecta con los requerimientos.  Son todas las actividades que se harán en el proyecto para dar cumplimiento a los requerimientos. Se recomienda mostrar el WBS mediante una lista indexada:

\begin{enumerate}
\item Grupo de tareas 1 (suma h)
	\begin{enumerate}
	\item Tarea 1 (tantas h)
	\item Tarea 2 (tantas h)
	\item Tarea 3 (tantas h)
	\end{enumerate}
\item Grupo de tareas 2 (suma h)
	\begin{enumerate}
	\item Tarea 1 (tantas h)
	\item Tarea 2 (tantas h)
	\item Tarea 3 (tantas h)
	\end{enumerate}
\item Grupo de tareas 3 (suma h)
	\begin{enumerate}
	\item Tarea 1 (tantas h)
	\item Tarea 2 (tantas h)
	\item Tarea 3 (tantas h)
	\item Tarea 4 (tantas h)
	\item Tarea 5 (tantas h)
	\end{enumerate}
\end{enumerate}

Cantidad total de horas: tantas.

\textbf{¡Importante!:} la unidad de horas es h y va separada por espacio del número. Es incorrecto escribir ``23hs".

\textbf{Se recomienda que no haya ninguna tarea que lleve más de 40 h.} De ser así se recomienda dividirla en tareas de menor duración.

\end{consigna}

\section{10. Diagrama de Activity On Node}
\label{sec:AoN}

\begin{consigna}{red}
Armar el AoN a partir del WBS definido en la etapa anterior.

Una herramienta simple para desarrollar los diagramas es el Draw.io (\url{https://app.diagrams.net/}).
\href{https://app.diagrams.net}{Draw.io}


\begin{figure}[htpb]
\centering 
\includegraphics[width=.8\textwidth]{./Figuras/AoN.png}
\caption{Diagrama de \textit{Activity on Node}.}
\label{fig:AoN}
\end{figure}

Indicar claramente en qué unidades están expresados los tiempos.
De ser necesario indicar los caminos semi críticos y analizar sus tiempos mediante un cuadro.
Es recomendable usar colores y un cuadro indicativo describiendo qué representa cada color.

\end{consigna}

\section{11. Diagrama de Gantt}
\label{sec:gantt}

\begin{consigna}{red}
Existen muchos programas y recursos \textit{online} para hacer diagramas de Gantt, entre los cuales destacamos:

\begin{itemize}
\item Planner
\item GanttProject
\item Trello + \textit{plugins}. En el siguiente link hay un tutorial oficial: \\ \url{https://blog.trello.com/es/diagrama-de-gantt-de-un-proyecto}
\item Creately, herramienta online colaborativa. \\\url{https://creately.com/diagram/example/ieb3p3ml/LaTeX}
\item Se puede hacer en latex con el paquete \textit{pgfgantt}\\ \url{http://ctan.dcc.uchile.cl/graphics/pgf/contrib/pgfgantt/pgfgantt.pdf}
\end{itemize}

Pegar acá una captura de pantalla del diagrama de Gantt, cuidando que la letra sea suficientemente grande como para ser legible. 
Si el diagrama queda demasiado ancho, se puede pegar primero la ``tabla'' del Gantt y luego pegar la parte del diagrama de barras del diagrama de Gantt.

Configurar el software para que en la parte de la tabla muestre los códigos del EDT (WBS).\\
Configurar el software para que al lado de cada barra muestre el nombre de cada tarea.\\
Revisar que la fecha de finalización coincida con lo indicado en el Acta Constitutiva.

En la figura \ref{fig:gantt}, se muestra un ejemplo de diagrama de gantt realizado con el paquete de \textit{pgfgantt}. 
En la plantilla pueden ver el código que lo genera y usarlo de base para construir el propio.

Las fechas pueden ser calculadas utilizando alguna de las herramientas antes citadas. Sin embargo, el siguiente ejemplo
fue elaborado utilizando 
\href{https://docs.google.com/spreadsheets/d/1fBz8NhSpc4tkkhz3KjJCbh1nR_ltDkfEcZi4tZXduqs}{esta hoja de cálculo}.

Es importante destacar que el ancho del diagrama estará dado por la longitud del texto utilizado para las tareas 
(Ejemplo: tarea 1, tarea 2, etcétera) y el valor \textit{x unit}. Para mejorar la apariencia del diagrama, es necesario
ajustar este valor y, quizás, acortar los nombres de las tareas.

\begin{figure}[htpb]
  \begin{center}
    \begin{ganttchart}[
      time slot unit=day,
      time slot format=isodate,
      x unit=0.038cm,
      y unit title=0.7cm,
      y unit chart=0.6cm,
      milestone/.append style={xscale=4}
      ]{2021-03-05}{2021-12-16}
      \gantttitlecalendar*{2021-03-05}{2021-12-16}{year} \\
      \gantttitlecalendar*{2021-03-05}{2021-12-16}{month} \\
      \ganttgroup{Duración Total}{2021-03-05}{2021-12-16} \\
      %%%%%%%%%%%%%%%%%Organización
      \ganttgroup{Organización}{2021-03-05}{2021-04-16} \\
      \ganttbar{Planificación del proyecto}{2021-03-05}{2021-04-15} \\
      %%%%%%%%%%%%%%%%%Ejecución
      \ganttgroup{Ejecución}{2021-04-16}{2021-10-21} \\
      \ganttbar{Tarea 1}{2021-04-16}{2021-04-29} \\
      \ganttbar{Tarea 2}{2021-04-30}{2021-05-13} \\
      \ganttbar{Tarea 3}{2021-05-14}{2021-05-27} \\
      \ganttbar{Tarea 4}{2021-05-28}{2021-07-12} \\
      \ganttbar{Tarea 5}{2021-07-13}{2021-08-09} \\
      \ganttbar{Tarea 6}{2021-08-10}{2021-09-23} \\
      \ganttbar{Tarea 7}{2021-09-24}{2021-09-30} \\
      \ganttbar{Tarea 8}{2021-10-01}{2021-10-14} \\
      \ganttbar{Tarea 9}{2021-10-15}{2021-10-21} \\
      % %%%%%%%%%%%%%%%%%Finalización
      \ganttgroup{Finalización}{2021-10-22}{2021-12-16} \\
      \ganttbar{Memoria v1}{2021-10-22}{2021-11-04} \\
      \ganttbar{Memoria v2}{2021-11-05}{2021-11-18} \\
      \ganttbar{Memoria final}{2021-11-19}{2021-12-02} \\
      % La fecha del siguiente milestone es la fecha en que terminamos la memoria
      \ganttmilestone{Enviar memoria al director}{2021-12-02} \\
      \ganttbar{Elaborar la presentación}{2021-12-03}{2021-12-16} \\
      \ganttmilestone{Ensayo de la presentación}{2021-12-16} \\
      %%%%%%%%%%%%%%%%%%%%%%%%%%%%%%%%%%%%%%%%%%%%%%%%%%%%%%%%%%%%%%%
    \end{ganttchart}
  \end{center}
  \caption{Diagrama de gantt de ejemplo}
  \label{fig:gantt}
\end{figure}


\begin{landscape}
\begin{figure}[htpb]
\centering 
\includegraphics[height=.85\textheight]{./Figuras/Gantt-2.png}
\caption{Ejemplo de diagrama de Gantt (apaisado).} %Modificar este título acorde.
\label{fig:diagGantt}
\end{figure}

\end{landscape}

\end{consigna}


\section{12. Presupuesto detallado del proyecto}
\label{sec:presupuesto}

\begin{consigna}{red}
Si el proyecto es complejo entonces separarlo en partes:
\begin{itemize}
	\item Un total global, indicando el subtotal acumulado por cada una de las áreas.
	\item El desglose detallado del subtotal de cada una de las áreas.
\end{itemize}

IMPORTANTE: No olvidarse de considerar los COSTOS INDIRECTOS.

Incluir la aclaración de si se emplea como moneda el peso argentino (ARS) o si se usa moneda extranjera (USD, EUR, etc). Si es en moneda extranjera se debe indicar la tasa de conversión respecto a la moneda local en una fecha dada.

\end{consigna}

\begin{table}[htpb]
\centering
\begin{tabularx}{\linewidth}{@{}|X|c|r|r|@{}}
\hline
\rowcolor[HTML]{C0C0C0} 
\multicolumn{4}{|c|}{\cellcolor[HTML]{C0C0C0}COSTOS DIRECTOS} \\ \hline
\rowcolor[HTML]{C0C0C0} 
Descripción &
  \multicolumn{1}{c|}{\cellcolor[HTML]{C0C0C0}Cantidad} &
  \multicolumn{1}{c|}{\cellcolor[HTML]{C0C0C0}Valor unitario} &
  \multicolumn{1}{c|}{\cellcolor[HTML]{C0C0C0}Valor total} \\ \hline
 &
  \multicolumn{1}{c|}{} &
  \multicolumn{1}{c|}{} &
  \multicolumn{1}{c|}{} \\ \hline
 &
  \multicolumn{1}{c|}{} &
  \multicolumn{1}{c|}{} &
  \multicolumn{1}{c|}{} \\ \hline
\multicolumn{1}{|l|}{} &
   &
   &
   \\ \hline
\multicolumn{1}{|l|}{} &
   &
   &
   \\ \hline
\multicolumn{3}{|c|}{SUBTOTAL} &
  \multicolumn{1}{c|}{} \\ \hline
\rowcolor[HTML]{C0C0C0} 
\multicolumn{4}{|c|}{\cellcolor[HTML]{C0C0C0}COSTOS INDIRECTOS} \\ \hline
\rowcolor[HTML]{C0C0C0} 
Descripción &
  \multicolumn{1}{c|}{\cellcolor[HTML]{C0C0C0}Cantidad} &
  \multicolumn{1}{c|}{\cellcolor[HTML]{C0C0C0}Valor unitario} &
  \multicolumn{1}{c|}{\cellcolor[HTML]{C0C0C0}Valor total} \\ \hline
\multicolumn{1}{|l|}{} &
   &
   &
   \\ \hline
\multicolumn{1}{|l|}{} &
   &
   &
   \\ \hline
\multicolumn{1}{|l|}{} &
   &
   &
   \\ \hline
\multicolumn{3}{|c|}{SUBTOTAL} &
  \multicolumn{1}{c|}{} \\ \hline
\rowcolor[HTML]{C0C0C0}
\multicolumn{3}{|c|}{TOTAL} &
   \\ \hline
\end{tabularx}%
\end{table}


\section{13. Gestión de riesgos}
\label{sec:riesgos}

\begin{consigna}{red}
a) Identificación de los riesgos (al menos cinco) y estimación de sus consecuencias:
 
Riesgo 1: detallar el riesgo (riesgo es algo que si ocurre altera los planes previstos de forma negativa)
\begin{itemize}
	\item Severidad (S): mientras más severo, más alto es el número (usar números del 1 al 10).\\
	Justificar el motivo por el cual se asigna determinado número de severidad (S).
	\item Probabilidad de ocurrencia (O): mientras más probable, más alto es el número (usar del 1 al 10).\\
	Justificar el motivo por el cual se asigna determinado número de (O). 
\end{itemize}   

Riesgo 2:
\begin{itemize}
	\item Severidad (S): X.\\
	Justificación...
	\item Ocurrencia (O): Y.\\
	Justificación...
\end{itemize}

Riesgo 3:
\begin{itemize}
	\item Severidad (S):  X.\\
	Justificación...
	\item Ocurrencia (O): Y.\\
	Justificación...
\end{itemize}


b) Tabla de gestión de riesgos:      (El RPN se calcula como RPN=SxO)

\begin{table}[htpb]
\centering
\begin{tabularx}{\linewidth}{@{}|X|c|c|c|c|c|c|@{}}
\hline
\rowcolor[HTML]{C0C0C0} 
Riesgo & S & O & RPN & S* & O* & RPN* \\ \hline
       &   &   &     &    &    &      \\ \hline
       &   &   &     &    &    &      \\ \hline
       &   &   &     &    &    &      \\ \hline
       &   &   &     &    &    &      \\ \hline
       &   &   &     &    &    &      \\ \hline
\end{tabularx}%
\end{table}

Criterio adoptado: 

Se tomarán medidas de mitigación en los riesgos cuyos números de RPN sean mayores a...

Nota: los valores marcados con (*) en la tabla corresponden luego de haber aplicado la mitigación.

c) Plan de mitigación de los riesgos que originalmente excedían el RPN máximo establecido:
 
Riesgo 1: plan de mitigación (si por el RPN fuera necesario elaborar un plan de mitigación).
  Nueva asignación de S y O, con su respectiva justificación:
  \begin{itemize}
	\item Severidad (S*): mientras más severo, más alto es el número (usar números del 1 al 10).
          Justificar el motivo por el cual se asigna determinado número de severidad (S).
	\item Probabilidad de ocurrencia (O*): mientras más probable, más alto es el número (usar del 1 al 10).
          Justificar el motivo por el cual se asigna determinado número de (O).
	\end{itemize}

Riesgo 2: plan de mitigación (si por el RPN fuera necesario elaborar un plan de mitigación).
 
Riesgo 3: plan de mitigación (si por el RPN fuera necesario elaborar un plan de mitigación).

\end{consigna}


\section{14. Gestión de la calidad}
\label{sec:calidad}

\begin{consigna}{red}
Elija al menos diez requerimientos que a su criterio sean los más importantes/críticos/que aportan más valor y para cada uno de ellos indique las acciones de verificación y validación que permitan asegurar su cumplimiento.

\begin{itemize} 
\item Req \#1: copiar acá el requerimiento con su correspondiente número.

\begin{itemize}
	\item Verificación para confirmar si se cumplió con lo requerido antes de mostrar el sistema al cliente. Detallar.
	\item Validación con el cliente para confirmar que está de acuerdo en que se cumplió con lo requerido. Detallar. 
\end{itemize}

\end{itemize}

Tener en cuenta que en este contexto se pueden mencionar simulaciones, cálculos, revisión de hojas de datos, consulta con expertos, mediciones, etc.  

Las acciones de verificación suelen considerar al entregable como ``caja blanca'', es decir se conoce en profundidad su funcionamiento interno.  

En cambio, las acciones de validación suelen considerar al entregable como ``caja negra'', es decir, que no se conocen los detalles de su funcionamiento interno.

\end{consigna}

\section{15. Procesos de cierre}    
\label{sec:cierre}

\begin{consigna}{red}
Establecer las pautas de trabajo para realizar una reunión final de evaluación del proyecto, tal que contemple las siguientes actividades:

\begin{itemize}
	\item Pautas de trabajo que se seguirán para analizar si se respetó el Plan de Proyecto original:\\
	 - Indicar quién se ocupará de hacer esto y cuál será el procedimiento a aplicar. 
	\item Identificación de las técnicas y procedimientos útiles e inútiles que se emplearon, los problemas que surgieron y cómo se solucionaron:\\
	 - Indicar quién se ocupará de hacer esto y cuál será el procedimiento para dejar registro.
	\item Indicar quién organizará el acto de agradecimiento a todos los interesados, y en especial al equipo de trabajo y colaboradores:\\
	  - Indicar esto y quién financiará los gastos correspondientes.
\end{itemize}

\end{consigna}

\end{document}